\documentclass[12pt,a4paper]{article}

\usepackage[UTF8]{ctex}
\usepackage{geometry}
\usepackage{amsmath}
\usepackage{amssymb}
\usepackage{enumitem}
\usepackage{xcolor}
\usepackage{titlesec}
\usepackage{fancyhdr}
\usepackage{setspace}
\usepackage[skins,breakable]{tcolorbox} % 引入高级盒子库
\usepackage{tikz}
\usepackage{fontawesome5}
\usepackage[colorlinks=true,linkcolor=secondarycolor,urlcolor=secondarycolor]{hyperref}

% --- 页面设置 ---
\geometry{left=2.5cm,right=2.5cm,top=2.8cm,bottom=2.8cm}
\setstretch{1.5} % 调整行间距

% --- 颜色定义 ---
\definecolor{maincolor}{RGB}{178, 34, 34}      % 深红色 (Firebrick)
\definecolor{secondarycolor}{RGB}{255, 127, 80} % 珊瑚色
\definecolor{accentcolor}{RGB}{139, 0, 0}      % 暗红色
\definecolor{bgcolor}{RGB}{250, 245, 240}      % 暖米色
\definecolor{boxbg}{RGB}{255, 252, 250}        % 极浅暖白
\definecolor{softblue}{RGB}{52,152,219}      % 软蓝色
\definecolor{gold}{RGB}{241,196,15}          % 金牌色


% --- 页眉页脚设置 ---
\pagestyle{fancy}
\fancyhf{}
\fancyhead[L]{\small\color{maincolor}\bfseries\faBook\ 毛概论述题复习资料}
\fancyhead[R]{\small\color{gray}\today}
\fancyfoot[C]{\small\color{gray}\thepage}

% 自定义页眉分割线
\renewcommand{\headrulewidth}{0pt}
\renewcommand{\footrulewidth}{0pt}
\fancyhead[C]{%
    \begin{tikzpicture}[remember picture, overlay]
        \draw[maincolor, line width=1.5pt] ([yshift=-1cm]current page.north west) -- ([yshift=-1cm]current page.north east);
    \end{tikzpicture}%
}

% --- 题目盒子样式 ---
\newtcolorbox{qbox}[2][]{
    enhanced,
    breakable,
    colback=boxbg,
    colframe=maincolor,
    coltitle=white,
    fonttitle=\bfseries\large,
    attach boxed title to top left={xshift=20pt, yshift*=-3.5mm},
    boxed title style={
        colback=maincolor,
        frame hidden,
        arc=3pt,
        drop lifted shadow=black!30
    },
    title={\faQuestionCircle\ 论述题 #2},
    drop fuzzy shadow,
    arc=5pt,
    left=15pt, right=15pt, top=18pt, bottom=12pt,
    before skip=25pt, after skip=10pt,
    #1
}

% --- 答案盒子样式 ---
\newtcolorbox{ansbox}{
    enhanced,              % 启用增强功能
    breakable,             % 允许盒子跨页断行
    frame hidden,          % 隐藏边框
    colback=white,         % 背景色为白色
    borderline west={4pt}{0pt}{accentcolor},  % 左侧边框:4pt宽,0pt偏移,使用accentcolor颜色
    left=15pt, right=10pt, top=5pt, bottom=5pt,  % 内边距设置
    coltext=black!85,      % 文本颜色为85%不透明度的黑色
    before skip=5pt, after skip=20pt  % 盒子前后的垂直间距
}

% --- 命令定义 ---
\newcommand{\question}[2]{
    \phantomsection
    \addcontentsline{toc}{section}{#1:#2}
    \begin{qbox}{#1}
        \color{maincolor!80!black}\bfseries #2
    \end{qbox}
}

\newcommand{\answer}[1]{
    \begin{ansbox}
        \textbf{\color{accentcolor}\faEdit\ 答:}#1
    \end{ansbox}
}

\newcommand{\reference}[1]{
    \par\vspace{0.8em}\hfill\textcolor{gold}{\small\faBookmark\ #1}
}

\newcommand{\highlight}[1]{
    \textcolor{softblue}{\textbf{#1}}
}

\setlist[enumerate]{label=\color{secondarycolor}\textbf{\arabic*.}, leftmargin=*, itemsep=0.6em}
\setlist[itemize]{label=\color{secondarycolor}\textbullet, leftmargin=*, itemsep=0.4em}

\begin{document}

% --- 封面页 ---
\newgeometry{margin=0cm}
\begin{titlepage}
    \begin{tikzpicture}[remember picture, overlay]
        % 1. 背景底色
        \fill[bgcolor] (current page.south west) rectangle (current page.north east);
        
        % 2. 背景纹理
        \begin{scope}
            \clip (current page.south west) rectangle (current page.north east);
            \foreach \x in {0,1,...,20} {
                \foreach \y in {0,1,...,29} {
                    \fill[maincolor, opacity=0.03] (\x*1.1, \y*1.1) circle (1pt);
                }
            }
        \end{scope}

        % 3. 顶部流线造型
        \fill[maincolor] (current page.north west) -- (current page.north east) -- ([yshift=-8cm]current page.north east) 
            .. controls ([yshift=-4cm, xshift=-4cm]current page.north east) and ([xshift=2cm]current page.center) .. 
            ([yshift=-3cm]current page.north west) -- cycle;
            
        \fill[secondarycolor, opacity=0.7] (current page.north west) -- ([xshift=12cm]current page.north west) 
            .. controls ([xshift=5cm, yshift=-5cm]current page.north west) and ([xshift=2cm, yshift=-2cm]current page.north west) .. 
            ([yshift=-10cm]current page.north west) -- cycle;

        % 4. 底部装饰
        \fill[maincolor, opacity=0.05] (current page.south west) circle (10cm);
        \node[maincolor, opacity=0.08, anchor=south east] at (current page.south east) {\fontsize{120}{120}\selectfont\faBookOpen};
        
        % 5. 顶部装饰图标
        \node[white, opacity=0.15, rotate=-15] at ([xshift=-4cm, yshift=-4cm]current page.north east) {\fontsize{80}{80}\selectfont\faPenNib};
    \end{tikzpicture}
    
    \centering
    \vspace*{8.5cm}
    
    % 副标题
    {\color{maincolor!80!black}\Large \faUniversity\ \textbf{期末复习资料系列}}\\[0.5cm]
    
    % 主标题框 (修改为透明背景)
    \begin{tcolorbox}[
        enhanced,
        frame hidden,       % 隐藏边框
        interior hidden,    % 隐藏内部填充(实现透明)
        coltext=maincolor,
        halign=center,
        overlay={
            % 背景大字水印 (保留水印,文字本身不会挡住背景)
            \node[anchor=south east, color=gray!10, font=\bfseries\fontsize{80}{80}\selectfont, xshift=2cm, yshift=-1cm] at (frame.south east) {MAO};
        }
    ]
        {\bfseries\fontsize{32}{42}\selectfont 毛泽东思想和中国特色\\[0.3em] 社会主义理论体系概论}
    \end{tcolorbox}
    
    \vspace{1.5cm}
    
    % 中间内容盒子
    \begin{tcolorbox}[
        enhanced,
        colback=white,
        colframe=maincolor,
        boxrule=0pt,
        leftrule=4pt,
        arc=0pt,
        width=0.85\textwidth,
        drop fuzzy shadow=gray!40,
        halign=left,
        left=1cm, right=1cm, top=1cm, bottom=1cm
    ]
        {\color{maincolor}\Large\bfseries \faList\ 课后论述题精编}\\[0.5em]
        \vspace{0.2cm}
        {\color{gray}\small \faCheckCircle\ 重点难点 \hfill \faCheckCircle\ 详细解析 \hfill \faCheckCircle\ 考前必看}
    \end{tcolorbox}
    
    \vfill
    
    % 底部信息
    \begin{tikzpicture}
        \node[align=center, text=gray!80!black] {
            {\large \faUser\ \textbf{整理:}计算01 \& 02}\\[0.8em]
            {\large \faCalendar\ \textbf{日期:}\today}
        };
    \end{tikzpicture}
    
    \vspace*{3cm}
\end{titlepage}
\restoregeometry

% --- 目录页 ---
\newpage
{
    \hypersetup{linkcolor=black}
    \begin{tcolorbox}[
        enhanced,
        colback=white,
        colframe=secondarycolor,
        title={\Large\bfseries\faListUl\ 目录},
        coltitle=white,
        colbacktitle=secondarycolor,
        arc=4pt,
        drop fuzzy shadow,
        top=10pt, bottom=10pt
    ]
        \renewcommand{\contentsname}{\vspace{-1.5cm}} 
        \tableofcontents
    \end{tcolorbox}
}

% --- 正文内容 ---

\question{1}{如何理解马克思主义中国化时代化的提出及其内涵要义?}

\question{2}{如何理解马克思主义中国化时代化两大理论成果及其历史地位?}

\question{3}{学习本课程的要求和方法是什么?}

\question{4}{为什么我们党能够开辟新民主主义革命道路、符合中国特点的社会主义改造道路、中国特色社会主义建设道路?}

\question{5}{如何理解新民主主义革命的政治、经济和文化纲领?}

\question{6}{如何理解新民主主义革命的基本经验?}

\question{7}{社会主义改造的历史经验是什么?}

\question{8}{社会主义建设道路初步探索过程中取得了哪些重大理论成果?}

\question{9}{如何认识中国特色社会主义理论体系的构成及其形成发展?}

\question{10}{如何认识邓小平理论的主要内容和历史地位?}

\question{11}{如何理解“三个代表”重要思想的核心观点?}

\question{12}{“三个代表”重要思想的主要内容包括哪些?}

\answer{
    “三个代表”重要思想的主要内容包括以下六点:

    \begin{enumerate}
        \item \highlight{发展是党执政兴国的第一要务}。必须始终紧紧抓住发展这个执政兴国的第一要务,将党的先进性和社会主义制度的优越性落实到发展先进生产力、发展先进文化、实现最广大人民的根本利益上来。发展是中国特色社会主义不断巩固和推进的关键。
        
        \item \highlight{建立社会主义市场经济体制}。在社会主义条件下发展市场经济是改革开放新的历史性突破。明确建立社会主义市场经济体制为我国经济体制改革的目标,既要发挥市场经济的长处,又要发挥社会主义制度的优越性,坚持从我国实际出发,走出一条自己的路。
        
        \item \highlight{全面建设小康社会}。立足于我国基本国情,在人民生活总体达到小康水平的基础上,提出全面建设更高水平、更全面、更平衡的小康社会的奋斗目标。
        
        \item \highlight{建设社会主义政治文明}。将建设社会主义政治文明与物质文明、精神文明一起确立为社会主义现代化全面发展的三大基本目标。最根本的是坚持党的领导、人民当家作主和依法治国的有机统一。
        
        \item \highlight{实施“引进来”和“走出去”相结合的对外开放战略}。对外开放是一项长期的基本国策。适应经济全球化趋势和加入世界贸易组织的新形势,坚持“引进来”和“走出去”相结合,全面提高对外开放水平。
        
        \item \highlight{推进党的建设新的伟大工程}。办好中国的事情关键在党,紧紧围绕建设什么样的党、怎样建设党的根本问题,以改革的精神加强和改进党的建设。
    \end{enumerate}
    
    \reference{参考:课本 P202}
}

\question{13}{如何理解科学发展观的科学内涵?}

\question{14}{如何理解社会主义核心价值体系是兴国之魂?}

\question{15}{如何理解构建社会主义和谐社会的内涵与要求?}

\question{16}{新世纪新阶段,如何提高党的建设的科学化水平?}

\end{document}
