\documentclass[12pt,a4paper]{article}

\usepackage[UTF8]{ctex}
\usepackage{geometry}
\usepackage{amsmath}
\usepackage{amssymb}
\usepackage{enumitem}
\usepackage{xcolor}
\usepackage{titlesec}
\usepackage{fancyhdr}
\usepackage{setspace}
\usepackage[skins,breakable]{tcolorbox} % 引入高级盒子库
\usepackage{tikz}
\usepackage{fontawesome5}
\usepackage[colorlinks=true,linkcolor=secondarycolor,urlcolor=secondarycolor]{hyperref}

% --- 页面设置 ---
\geometry{left=2.5cm,right=2.5cm,top=2.8cm,bottom=2.8cm}
\setstretch{1.5} % 调整行间距

% --- 颜色定义 (Flat UI 风格) ---
\definecolor{maincolor}{RGB}{44, 62, 80}       % 深蓝灰 (用于主标题、边框)
\definecolor{secondarycolor}{RGB}{41, 128, 185}% 亮蓝 (用于链接、次要元素)
\definecolor{accentcolor}{RGB}{192, 57, 43}    % 砖红 (用于强调、答案标记)
\definecolor{bgcolor}{RGB}{236, 240, 241}      % 浅灰 (用于背景)
\definecolor{boxbg}{RGB}{250, 250, 250}        % 极浅灰 (用于盒子背景)

% --- 页眉页脚设置 ---
\pagestyle{fancy}
\fancyhf{}
\fancyhead[L]{\small\color{maincolor}\bfseries\faBook\ 毛概论述题复习资料}
\fancyhead[R]{\small\color{gray}\today}
\fancyfoot[C]{\small\color{gray}\thepage}

% 自定义页眉分割线 (TikZ 绘制)
\renewcommand{\headrulewidth}{0pt}
\renewcommand{\footrulewidth}{0pt}
\fancyhead[C]{%
    \begin{tikzpicture}[remember picture, overlay]
        \draw[maincolor, line width=1.5pt] ([yshift=-1cm]current page.north west) -- ([yshift=-1cm]current page.north east);
    \end{tikzpicture}%
}

% --- 题目盒子样式 ---
\newtcolorbox{qbox}[2][]{
    enhanced,           % 启用高级绘图引擎
    breakable,          % 允许跨页
    colback=boxbg,      % 背景色
    colframe=maincolor, % 边框色
    coltitle=white,     % 标题文字颜色
    fonttitle=\bfseries\large,
    attach boxed title to top left={xshift=20pt, yshift*=-3.5mm}, % 标题位置
    boxed title style={
        colback=maincolor,
        frame hidden,
        arc=3pt,
        drop lifted shadow=black!30 % 标题阴影
    },
    title={\faQuestionCircle\ 论述题 #2},
    drop fuzzy shadow,  % 盒子整体阴影
    arc=5pt,
    left=15pt,
    right=15pt,
    top=18pt,           % 顶部留白给标题
    bottom=12pt,
    before skip=25pt,   % 盒子前间距
    after skip=10pt,    % 盒子后间距
    #1
}

% --- 答案盒子样式 ---
\newtcolorbox{ansbox}{
    enhanced,
    breakable,
    frame hidden,       % 隐藏四周线框
    colback=white,      % 背景纯白
    borderline west={4pt}{0pt}{accentcolor}, % 左侧加粗装饰线
    left=15pt,
    right=10pt,
    top=5pt,
    bottom=5pt,
    coltext=black!85,   % 正文颜色略深
    before skip=5pt,
    after skip=20pt
}

% --- 命令定义 ---

\newcommand{\question}[2]{
    \phantomsection
    \addcontentsline{toc}{section}{#1:#2}
    \begin{qbox}{#1}
        \color{maincolor!80!black}\bfseries #2
    \end{qbox}
}

\newcommand{\answer}[1]{
    \begin{ansbox}
        \textbf{\color{accentcolor}\faEdit\ 答:}#1
    \end{ansbox}
}

\newcommand{\reference}[1]{
    \par\vspace{0.8em}\hfill\textcolor{secondarycolor}{\small\faBookmark\ #1}
}

\newcommand{\highlight}[1]{
    \textcolor{accentcolor}{\textbf{#1}}
}

% 列表样式优化
\setlist[enumerate]{label=\color{secondarycolor}\textbf{\arabic*.}, leftmargin=*, itemsep=0.6em}
\setlist[itemize]{label=\color{secondarycolor}\textbullet, leftmargin=*, itemsep=0.4em}

\begin{document}

% --- 封面页 ---
\begin{titlepage}
    \begin{tikzpicture}[remember picture, overlay]
        % 背景装饰
        \fill[bgcolor] (current page.south west) rectangle (current page.north east);
        % 角落圆
        \fill[maincolor] (current page.north west) circle (5cm);
        \fill[secondarycolor] (current page.south east) circle (6cm);
        % 图标水印
        \node[anchor=center, opacity=0.03, rotate=30] at (current page.center) {\fontsize{150}{150}\selectfont\faBook};
    \end{tikzpicture}
    
    \centering
    \vspace*{4cm}
    
    {\color{secondarycolor}\Large \faUniversity\ 复习资料整理}\\[0.8cm]
    
    {\color{maincolor}\bfseries\fontsize{30}{38}\selectfont 毛泽东思想和中国特色社会主义\\[0.3em] 理论体系概论}\\[2cm]
    
    % 封面中间的盒子
    \begin{tcolorbox}[
        enhanced,
        colback=white,
        colframe=maincolor,
        boxrule=1.5pt,
        arc=6pt,
        width=0.85\textwidth,
        drop fuzzy shadow,
        halign=center,
        top=1.2cm, bottom=1.2cm
    ]
        {\Large\bfseries \faList\ 论述题专项整理}\\[0.8em]
        {\color{gray}\small 重点难点 \textbullet\ 详细解析 \textbullet\ 考前必看}
    \end{tcolorbox}
    
    \vfill
    
    {\color{maincolor}\large \textbf{\faUser\ 整理:}计算01 \& 02}\\[0.8em]
    {\color{maincolor}\large \textbf{\faCalendar\ 日期:}\today}
    
    \vspace*{2.5cm}
\end{titlepage}

% --- 目录页 ---
\newpage
{
    \hypersetup{linkcolor=black} % 目录链接黑色,避免花哨
    \begin{tcolorbox}[
        enhanced,
        colback=white,
        colframe=secondarycolor,
        title={\Large\bfseries\faListUl\ 目录},
        coltitle=white,
        colbacktitle=secondarycolor,
        arc=4pt,
        drop fuzzy shadow,
        top=10pt, bottom=10pt
    ]
        % 隐藏目录自身的标题(因为盒子已经有标题了)
        \renewcommand{\contentsname}{\vspace{-1.5cm}} 
        \tableofcontents
    \end{tcolorbox}
}
\newpage

% --- 正文内容 ---

\question{1}{如何理解马克思主义中国化时代化的提出及其内涵要义?}

\question{2}{如何理解马克思主义中国化时代化两大理论成果及其历史地位?}

\question{3}{学习本课程的要求和方法是什么?}

\question{4}{为什么我们党能够开辟新民主主义革命道路、符合中国特点的社会主义改造道路、中国特色社会主义建设道路?}

\question{5}{如何理解新民主主义革命的政治、经济和文化纲领?}

\question{6}{如何理解新民主主义革命的基本经验?}

\question{7}{社会主义改造的历史经验是什么?}

\question{8}{社会主义建设道路初步探索过程中取得了哪些重大理论成果?}

\question{9}{如何认识中国特色社会主义理论体系的构成及其形成发展?}

\question{10}{如何认识邓小平理论的主要内容和历史地位?}

\question{11}{如何理解“三个代表”重要思想的核心观点?}

\question{12}{“三个代表”重要思想的主要内容包括哪些?}

\answer{
    “三个代表”重要思想的主要内容包括以下六点:

    \begin{enumerate}
        \item \highlight{发展是党执政兴国的第一要务}。必须始终紧紧抓住发展这个执政兴国的第一要务,将党的先进性和社会主义制度的优越性落实到发展先进生产力、发展先进文化、实现最广大人民的根本利益上来。发展是中国特色社会主义不断巩固和推进的关键。
        
        \item \highlight{建立社会主义市场经济体制}。在社会主义条件下发展市场经济是改革开放新的历史性突破。明确建立社会主义市场经济体制为我国经济体制改革的目标,既要发挥市场经济的长处,又要发挥社会主义制度的优越性,坚持从我国实际出发,走出一条自己的路。
        
        \item \highlight{全面建设小康社会}。立足于我国基本国情,在人民生活总体达到小康水平的基础上,提出全面建设更高水平、更全面、更平衡的小康社会的奋斗目标。
        
        \item \highlight{建设社会主义政治文明}。将建设社会主义政治文明与物质文明、精神文明一起确立为社会主义现代化全面发展的三大基本目标。最根本的是坚持党的领导、人民当家作主和依法治国的有机统一。
        
        \item \highlight{实施“引进来”和“走出去”相结合的对外开放战略}。对外开放是一项长期的基本国策。适应经济全球化趋势和加入世界贸易组织的新形势,坚持“引进来”和“走出去”相结合,全面提高对外开放水平。
        
        \item \highlight{推进党的建设新的伟大工程}。办好中国的事情关键在党,紧紧围绕建设什么样的党、怎样建设党的根本问题,以改革的精神加强和改进党的建设。
    \end{enumerate}
    
    \reference{参考:课本 P202}
}

\question{13}{如何理解科学发展观的科学内涵?}

\question{14}{如何理解社会主义核心价值体系是兴国之魂?}

\question{15}{如何理解构建社会主义和谐社会的内涵与要求?}

\question{16}{新世纪新阶段,如何提高党的建设的科学化水平?}

\end{document}
