\documentclass[12pt,a4paper]{article}

\usepackage[UTF8]{ctex}
\usepackage{geometry}
\usepackage{amsmath}
\usepackage{amssymb}
\usepackage{enumitem}
\usepackage{xcolor}
\usepackage{titlesec}
\usepackage{fancyhdr}
\usepackage{setspace}
\usepackage{tcolorbox}
\usepackage{tikz}
\usepackage{fontawesome5}
\usepackage{hyperref}

\geometry{left=2.2cm,right=2.2cm,top=2.5cm,bottom=2.5cm}

\definecolor{maincolor}{RGB}{26,60,97}
\definecolor{accentcolor}{RGB}{192,57,43}
\definecolor{secondarycolor}{RGB}{41,128,185}
\definecolor{lightgray}{RGB}{236,240,241}
\definecolor{darkgray}{RGB}{52,73,94}
\definecolor{gold}{RGB}{241,196,15}
\definecolor{softblue}{RGB}{52,152,219}

\titleformat{\section}
{\Large\bfseries\color{maincolor}}
{\thesection}{1em}{}[{\titlerule[0.8pt]}]

\titleformat{\subsection}
{\large\bfseries\color{secondarycolor}}
{\thesubsection}{1em}{}

\setlist[itemize]{leftmargin=2em,itemsep=0.4em}
\setlist[enumerate]{leftmargin=2em,itemsep=0.5em,label=\color{maincolor}\textbf{\arabic*.}}

\pagestyle{fancy}
\fancyhf{}
\fancyhead[L]{\small\color{darkgray}论述题整理}
\fancyhead[R]{\small\color{darkgray}\today}
\fancyfoot[L]{\small\color{gray}计算01 \& 02}
\fancyfoot[R]{\small\color{gray}\thepage}

\renewcommand{\headrulewidth}{0.8pt}
\renewcommand{\headrule}{{\color{maincolor}\hrule height\headrulewidth}}
\renewcommand{\footrulewidth}{0.5pt}
\renewcommand{\footrule}{{\color{lightgray}\hrule height\footrulewidth}}

\setstretch{1.4}

\newcommand{\question}[2]{
    \vspace{0.8em}
    \begin{tcolorbox}[
        colback=lightgray!30,
        colframe=maincolor,
        boxrule=1.2pt,
        arc=3pt,
        left=8pt,
        right=8pt,
        top=6pt,
        bottom=6pt,
        title={\small\color{white}\faQuestionCircle\ 论述题 #1},
        coltitle=white,
        colbacktitle=maincolor
    ]
        \textbf{\large\color{darkgray}#2}
    \end{tcolorbox}
    \vspace{0.3em}
}

\newcommand{\answer}[1]{
    \vspace{0.5em}
    % 关键修改:去掉了 minipage 环境,允许内容自然跨页
    \noindent\textbf{\color{accentcolor}\faEdit\ 答:}#1
    \vspace{0.8em}
}


\newcommand{\reference}[1]{
    \textcolor{accentcolor}{\small\faBookmark\ (#1)}
}

\newcommand{\highlight}[1]{
    \textcolor{maincolor}{\textbf{#1}}
}

\begin{document}

\begin{center}
  
    % 主标题
    {\color{maincolor}\LARGE\bfseries 《毛泽东思想和中国特色社会主义理论体系概论》}\\[0.8em]
    
    % 副标题
    {\color{secondarycolor}\Large \faBook\ 论述题复习}\\[0.8em]
    
    % 装饰线条
    {\color{maincolor}\hrule height 2pt}
\end{center}






\question{12}{“三个代表”重要思想的主要内容包括哪些?}

\answer{\reference{课本P202}“三个代表”重要思想的主要内容包括以下六点:

\begin{enumerate}
    \item \highlight{发展是党执政兴国的第一要务}。必须始终紧紧抓住发展这个执政兴国的第一要务,将党的先进性和社会主义制度的优越性落实到发展先进生产力、发展先进文化、实现最广大人民的根本利益上来。发展是中国特色社会主义不断巩固和推进的关键,也是在国际竞争中赢得主动、解决前进中问题的根本途径。

    \item \highlight{建立社会主义市场经济体制}。在社会主义条件下发展市场经济是改革开放新的历史性突破,是前无古人的伟大创举。明确建立社会主义市场经济体制为我国经济体制改革的目标,既要发挥市场经济的长处,又要发挥社会主义制度的优越性,坚持从我国实际出发,走出一条自己的路。

    \item \highlight{全面建设小康社会}。立足于我国基本国情,在人民生活总体达到小康水平的基础上,提出全面建设更高水平、更全面、更平衡的小康社会的奋斗目标。针对当时小康低水平、不全面、不平衡的问题,进行了前瞻性的战略规划。

    \item \highlight{建设社会主义政治文明}。将建设社会主义政治文明与物质文明、精神文明一起确立为社会主义现代化全面发展的三大基本目标。最根本的是坚持党的领导、人民当家作主和依法治国的有机统一,发展社会主义民主政治,建设社会主义法治国家。

    \item \highlight{实施“引进来”和“走出去”相结合的对外开放战略}。对外开放是一项长期的基本国策。适应经济全球化趋势和加入世界贸易组织的新形势,坚持"引进来"和"走出去"相结合,全面提高对外开放水平,充分利用国际国内两个市场、两种资源,以开放促改革促发展。

    \item \highlight{推进党的建设新的伟大工程}。办好中国的事情关键在党,紧紧围绕建设什么样的党、怎样建设党的根本问题,以改革的精神加强和改进党的建设,坚持党的领导核心地位,增强党的阶级基础和扩大党的群众基础,使党始终保持先进性和纯洁性。
\end{enumerate}
}
\end{document}


